\chapter{Strumenti per grafici, immagini, etc.}

Questo capitolo contiene una breve panoramica sugli strumenti consigliati per 
la gestione di grafici, immagini etc.

\section{Grafici}

Per  grafici  2D il  prodotto  consigliato  è \texttt{xmgrace}  che  trovate
installato su  alcune macchine. Il suo  punto di forza è  la semplicità, in
pratica gli  date in pasto  un file ASCII contenente  tutti i valori  e poi
gestite il contorno.

\section{Immagini}

Un qualunque prodotto fra i seguenti:
\begin{description}
\item[xv:] fondamentalmente permette di salvare immagini già esistenti in differenti formati e, al limite, di
selezionare una sottoimmagine o cambiare i colori. È insostituibile quando 
occorre acquisire l'output di una finestra sotto X11; in questo caso mediante il pulsante grab è possibile
decidere se acquisire una finestra o anche tutto lo schermo.
\item[gimp:] programma abbastanza sofisticato di fotoritocco.
\item[xpaint:] come il precedente ma molto più limitato; però anche molto più veloce.
\end{description}

\section{Correzzione ortografica}

Sì lo so, c'è un errore~:-).
L'unico programma disponibile è \texttt{aspell}. Funziona bene se si ha avuto l'accortezza di usare dirrettamente le lettere
accentate.
Si lancia con: \texttt{aspell --lang=it --mode=tex check <elenco file da controllare>}, dopodiché si ferma su ogni parola considerata
sospetta, mostrando eventuali suggerimenti. I comandi principali sono:
(A)~accetta la parola (anche le successive occorrenze), (I)~inserisci la parola nel vocabolario personale, (R)~reinserisci la parola e
(numero)~sostituisci la parola con quella suggerita avente numero nn.




